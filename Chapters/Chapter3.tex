%--------------------------
%-------------------------------------------

\chapter{The Proposed Solution}


This chapter presents a detailed description of the problem and the proposed solution. We have carried out
an experiment to study the power dissipation problem for different set of applications on two different
processor architecture. The problem we identified is the excessive power dissipation in the Execution unit
of a core. We addressed this problem by proposing a solution which provides a provision for turning-off and
turning-on the selected functional units dynamically at any given point of time.


%\section{Power Dissipation in Execution Unit}
% The experiment on power dissipation in Execution unit 




\section{The Architecture}

\begin{figure}
 \begin{center} 
  \includegraphics[scale=0.5]{superscalarcore.jpg}
  \caption{A core with super scalar processor architecture}
\end{center}
\end{figure}

%
%
%

\begin{figure}
 \begin{center}   
  \includegraphics[scale=0.5]{superscalar-withpmu.jpg}
  \caption{A core with super scalar processor architecture with a proposed power reducing unit}
 \end{center}
\end{figure}

%
%
%

\begin{figure}
   \begin{center}
     \includegraphics[scale=1]{counter.pdf}
     \caption{A unit to track the threshold for deciding the power-off based on counter value}
   \end{center}
\end{figure}


\section{Working Principle}

%------------------------------------------------------------------------
