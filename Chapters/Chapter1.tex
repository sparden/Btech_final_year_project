\chapter{INTRODUCTION}
\label{chap:intro}

%\section{Abstract}


\section{Introduction and Background}

Although the performance level of out-of-order superscalar processors is high,they consume much more energy
than do in-order superscalar processors,because a large amount of energy is consumed by hardware for dynamic
instruction scheduling such as an instruction queue(IQ) and a load/store queue (LSQ), which comprises mainly
heavily multi-ported memories.

The energy consumption per access of multi-ported memory. The energy consumption per access of multi-ported
memory is proportional to its capacity and the number of its ports. Moreover, the number of accesses is also
increased with the issue width. Consequently, its energy consumption is very large. We are going to develop
a power-aware Adaptive superscalar Architecture which consumes lower energy and saves more power than the
existed architectures.

%
%

\section{Power Dissipation}

In this section we will provide a background on power dissipation. There are two type of power
dissipations \cite{power-dissipation}: 

\begin{itemize}
   \item Dynamic power dissipation \\

          \begin{itemize}
               \item Glitch power
               \item Functional power
          \end{itemize}
   \item Static power dissipation        
          
\end{itemize}

The dynamic power is defined as the switching power that is dissipated when the processor is in operation.
Mathematically, the dynamic power is computed as the total energy divided by a given time duration. The
dynamic power comes mainly from two sources: \textit{true functional activity} and \textit{glitch related
activities}. The true functional activity which is required for the computation is the compulsory power that
the processor has to dissipate in order to perform the computational task. Whereas the glitch activity are
not the essential activity but they comes due to inherent limitation in circuit design. In this work, our
objective is to find out the architectural level solutions to reduce the total power dissipation by reducing
the dynamic glitch power. 

The other type of power dissipation that has gain importance is the static power. The static power is also
known as leakage because it happens irrespective of any change in inputs. So far there is power supply on,
the leakage power continue to dissipate.  


\section{The Power Dissipation in Multi-core Architecture}

The excessive power dissipation puts a limit on performance improvement of processor.
We address the dynamic power problem by exploring the possibility of reducing the glitch activity in
Execution unit of superscalar processor core. 

We have carried out a set of experiment to evaluate the power dissipation for difference benchmark from
PARSEC and SPLASH suit. The experiment has been carried out to understand and observe the power dissipation
profile of each benchmark program for different configuration of cores. Further, the experiment has also
been carried out to observe the power dissipation of each units at micro-architecture level. From the
plot it can be observed that the power dissipation in Execution unit which are labaled in the plots as
Core-fp, Core-ALU, and Core-int are relatively high compared to the other stages of super-scalar pipeline.

Also from the plot of total power dissipation for entire core it can be observed that the power dissipation
for different benchmark vary differently. This observation leads us to the understanding that to reduce the
over all power dissipation by some means the problem needs to addressed on the basis of application
(application here we mean the benchmark program). Therefore we looked for the solution which will be
addressing the power problem on the basis of application. Which leads us to the solution which is of dynamic
in nature. 

\begin{figure}
   \includegraphics[scale=1]{./Pictures/PARSEC_neh.jpg}
   \caption{Total power dissipation for PARSEC benchmark with varying number of cores}
\end{figure}

%
%
%

\begin{figure}
   \includegraphics[scale=1]{./Pictures/PARSEC_neh.jpg}
   \caption{Total power dissipation for SPLASH benchmark with varying number of cores}
\end{figure}

%
%
%

\section{The Contributions}

\begin{itemize}

   \item Experiments to identify the power dissipation problem in Execution unit

   \item Designing a configuration for atom-like processor core
   
   \item Architecture for reducing glitch activity in Execution unit

\end{itemize}

\section{Chapter Organisation}

The Chapter II provides literature review on power dissipation problem in multi-core architecture. The
Chapter III explains in detail the problem of interest and the proposed solution. The Chapter IV gives
experimental results and report concludes in Chapter V. Additional information on sniper simulator, 
bench marks, and few programs are provided in Appendix. 




